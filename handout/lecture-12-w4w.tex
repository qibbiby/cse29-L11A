\documentclass[12pt, oneside,landscape ]{article}
\usepackage{fontspec}

\usepackage{emoji}

\setmainfont{DejaVu Serif}
\setsansfont{DejaVu Sans}
\setmonofont{DejaVu Sans Mono}

\newfontfamily\HiraginoSans{Hiragino Sans}


\include{header}

\usepackage{verbatim}

\begin{document}

\begin{minipage}{0.45\textwidth}
{\scriptsize
In contrast to C, languages like Python and Java (and others) have {\em
immutable} strings that support operations like concatenation with {\tt +}. How
does that work in the machine?}
\begin{shell}[basicstyle=\ttfamily\scriptsize]
$ python3
>>> a = "hello "
>>> b = "cse29"
>>> c = a + b
>>> c
'hello cse29'
>>> a
'hello '
>>> b
'cse29'
\end{shell}
\end{minipage}
\hspace{2em}
\begin{minipage}{0.45\textwidth}
\begin{shell}[basicstyle=\ttfamily\scriptsize]
$ jshell
jshell> String a = "hello ";
a ==> "hello "
jshell> String b = "cse29";
b ==> "cse29"
jshell> String c = a + b;
c ==> "hello cse29"
jshell> a
a ==> "hello "
jshell> b
b ==> "cse29"
jshell> c
c ==> "hello cse29"
\end{shell}
\end{minipage}


\begin{lstlisting}[basicstyle=\ttfamily\scriptsize]
#include <stdint.h>
#include <stdio.h>
#include <stdlib.h>
#include <string.h>
#include <assert.h>

struct String {
    uint64_t length; // should always equal strlen(contents)
    char* contents;  // should always have allocated space of length + 1
};

typedef struct String String;

String new_String(char* init_contents) {
    uint64_t size = strlen(init_contents);
    String r = { size, init_contents };
    return r;
}

String plus(String s1, String s2) {
    uint64_t new_size = s1.length + s2.length + 1;
    char new_contents[new_size];
    strncpy(new_contents, s1.contents, s1.length);
    strncpy(new_contents + s1.length, s2.contents, s2.length);
    new_contents[new_size - 1] = 0;
    String r = { new_size - 1, new_contents };
    return r;
}

int main() {
    String s = new_String("hello");
    printf("%s\n", s.contents);

    String s2 = new_String("cse29");

    String hello_cse = plus(s, s2);
    String hello_bang = plus(s, new_String("!!!!"));

    printf("%s\n", hello_cse.contents);
    printf("%s\n", hello_bang.contents);
}
\end{lstlisting}

\newpage

\begin{lstlisting}
String plus_heap(String s1, String s2) {
    uint64_t new_size = s1.length + s2.length + 1;
    char* new_contents = calloc(new_size, sizeof(char));
    strncpy(new_contents, s1.contents, s1.length);
    strncpy(new_contents + s1.length, s2.contents, s2.length);
    new_contents[new_size - 1] = 0;
    String r = { new_size, new_contents };
    return r;
}
int main() {
    String s = new_String("hello");
    printf("%s\n", s.contents);

    String s2 = new_String("cse29");

    String hello_cse = plus_heap(s, s2);
    String hello_bang = plus_heap(s, new_String("!!!!"));

    printf("%s\n", hello_cse.contents);
    printf("%s\n", hello_bang.contents);
}
\end{lstlisting}

\end{document}
